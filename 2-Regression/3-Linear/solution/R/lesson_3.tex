% Options for packages loaded elsewhere
\PassOptionsToPackage{unicode}{hyperref}
\PassOptionsToPackage{hyphens}{url}
%
\documentclass[
]{article}
\usepackage{amsmath,amssymb}
\usepackage{lmodern}
\usepackage{ifxetex,ifluatex}
\ifnum 0\ifxetex 1\fi\ifluatex 1\fi=0 % if pdftex
  \usepackage[T1]{fontenc}
  \usepackage[utf8]{inputenc}
  \usepackage{textcomp} % provide euro and other symbols
\else % if luatex or xetex
  \usepackage{unicode-math}
  \defaultfontfeatures{Scale=MatchLowercase}
  \defaultfontfeatures[\rmfamily]{Ligatures=TeX,Scale=1}
\fi
% Use upquote if available, for straight quotes in verbatim environments
\IfFileExists{upquote.sty}{\usepackage{upquote}}{}
\IfFileExists{microtype.sty}{% use microtype if available
  \usepackage[]{microtype}
  \UseMicrotypeSet[protrusion]{basicmath} % disable protrusion for tt fonts
}{}
\makeatletter
\@ifundefined{KOMAClassName}{% if non-KOMA class
  \IfFileExists{parskip.sty}{%
    \usepackage{parskip}
  }{% else
    \setlength{\parindent}{0pt}
    \setlength{\parskip}{6pt plus 2pt minus 1pt}}
}{% if KOMA class
  \KOMAoptions{parskip=half}}
\makeatother
\usepackage{xcolor}
\IfFileExists{xurl.sty}{\usepackage{xurl}}{} % add URL line breaks if available
\IfFileExists{bookmark.sty}{\usepackage{bookmark}}{\usepackage{hyperref}}
\hypersetup{
  pdftitle={Build a regression model: linear and polynomial regression models},
  hidelinks,
  pdfcreator={LaTeX via pandoc}}
\urlstyle{same} % disable monospaced font for URLs
\usepackage[margin=1in]{geometry}
\usepackage{color}
\usepackage{fancyvrb}
\newcommand{\VerbBar}{|}
\newcommand{\VERB}{\Verb[commandchars=\\\{\}]}
\DefineVerbatimEnvironment{Highlighting}{Verbatim}{commandchars=\\\{\}}
% Add ',fontsize=\small' for more characters per line
\usepackage{framed}
\definecolor{shadecolor}{RGB}{248,248,248}
\newenvironment{Shaded}{\begin{snugshade}}{\end{snugshade}}
\newcommand{\AlertTok}[1]{\textcolor[rgb]{0.94,0.16,0.16}{#1}}
\newcommand{\AnnotationTok}[1]{\textcolor[rgb]{0.56,0.35,0.01}{\textbf{\textit{#1}}}}
\newcommand{\AttributeTok}[1]{\textcolor[rgb]{0.77,0.63,0.00}{#1}}
\newcommand{\BaseNTok}[1]{\textcolor[rgb]{0.00,0.00,0.81}{#1}}
\newcommand{\BuiltInTok}[1]{#1}
\newcommand{\CharTok}[1]{\textcolor[rgb]{0.31,0.60,0.02}{#1}}
\newcommand{\CommentTok}[1]{\textcolor[rgb]{0.56,0.35,0.01}{\textit{#1}}}
\newcommand{\CommentVarTok}[1]{\textcolor[rgb]{0.56,0.35,0.01}{\textbf{\textit{#1}}}}
\newcommand{\ConstantTok}[1]{\textcolor[rgb]{0.00,0.00,0.00}{#1}}
\newcommand{\ControlFlowTok}[1]{\textcolor[rgb]{0.13,0.29,0.53}{\textbf{#1}}}
\newcommand{\DataTypeTok}[1]{\textcolor[rgb]{0.13,0.29,0.53}{#1}}
\newcommand{\DecValTok}[1]{\textcolor[rgb]{0.00,0.00,0.81}{#1}}
\newcommand{\DocumentationTok}[1]{\textcolor[rgb]{0.56,0.35,0.01}{\textbf{\textit{#1}}}}
\newcommand{\ErrorTok}[1]{\textcolor[rgb]{0.64,0.00,0.00}{\textbf{#1}}}
\newcommand{\ExtensionTok}[1]{#1}
\newcommand{\FloatTok}[1]{\textcolor[rgb]{0.00,0.00,0.81}{#1}}
\newcommand{\FunctionTok}[1]{\textcolor[rgb]{0.00,0.00,0.00}{#1}}
\newcommand{\ImportTok}[1]{#1}
\newcommand{\InformationTok}[1]{\textcolor[rgb]{0.56,0.35,0.01}{\textbf{\textit{#1}}}}
\newcommand{\KeywordTok}[1]{\textcolor[rgb]{0.13,0.29,0.53}{\textbf{#1}}}
\newcommand{\NormalTok}[1]{#1}
\newcommand{\OperatorTok}[1]{\textcolor[rgb]{0.81,0.36,0.00}{\textbf{#1}}}
\newcommand{\OtherTok}[1]{\textcolor[rgb]{0.56,0.35,0.01}{#1}}
\newcommand{\PreprocessorTok}[1]{\textcolor[rgb]{0.56,0.35,0.01}{\textit{#1}}}
\newcommand{\RegionMarkerTok}[1]{#1}
\newcommand{\SpecialCharTok}[1]{\textcolor[rgb]{0.00,0.00,0.00}{#1}}
\newcommand{\SpecialStringTok}[1]{\textcolor[rgb]{0.31,0.60,0.02}{#1}}
\newcommand{\StringTok}[1]{\textcolor[rgb]{0.31,0.60,0.02}{#1}}
\newcommand{\VariableTok}[1]{\textcolor[rgb]{0.00,0.00,0.00}{#1}}
\newcommand{\VerbatimStringTok}[1]{\textcolor[rgb]{0.31,0.60,0.02}{#1}}
\newcommand{\WarningTok}[1]{\textcolor[rgb]{0.56,0.35,0.01}{\textbf{\textit{#1}}}}
\usepackage{longtable,booktabs,array}
\usepackage{calc} % for calculating minipage widths
% Correct order of tables after \paragraph or \subparagraph
\usepackage{etoolbox}
\makeatletter
\patchcmd\longtable{\par}{\if@noskipsec\mbox{}\fi\par}{}{}
\makeatother
% Allow footnotes in longtable head/foot
\IfFileExists{footnotehyper.sty}{\usepackage{footnotehyper}}{\usepackage{footnote}}
\makesavenoteenv{longtable}
\usepackage{graphicx}
\makeatletter
\def\maxwidth{\ifdim\Gin@nat@width>\linewidth\linewidth\else\Gin@nat@width\fi}
\def\maxheight{\ifdim\Gin@nat@height>\textheight\textheight\else\Gin@nat@height\fi}
\makeatother
% Scale images if necessary, so that they will not overflow the page
% margins by default, and it is still possible to overwrite the defaults
% using explicit options in \includegraphics[width, height, ...]{}
\setkeys{Gin}{width=\maxwidth,height=\maxheight,keepaspectratio}
% Set default figure placement to htbp
\makeatletter
\def\fps@figure{htbp}
\makeatother
\setlength{\emergencystretch}{3em} % prevent overfull lines
\providecommand{\tightlist}{%
  \setlength{\itemsep}{0pt}\setlength{\parskip}{0pt}}
\setcounter{secnumdepth}{-\maxdimen} % remove section numbering
\ifluatex
  \usepackage{selnolig}  % disable illegal ligatures
\fi

\title{Build a regression model: linear and polynomial regression
models}
\author{}
\date{\vspace{-2.5em}}

\begin{document}
\maketitle

{
\setcounter{tocdepth}{2}
\tableofcontents
}
\hypertarget{linear-and-polynomial-regression-for-pumpkin-pricing---lesson-3}{%
\subsection{Linear and Polynomial Regression for Pumpkin Pricing -
Lesson
3}\label{linear-and-polynomial-regression-for-pumpkin-pricing---lesson-3}}

\begin{figure}
\centering
\includegraphics[width=8.33333in,height=\textheight]{../../images/linear-polynomial.png}
\caption{Infographic by Dasani Madipalli}
\end{figure}

\hypertarget{introduction}{%
\paragraph{Introduction}\label{introduction}}

So far you have explored what regression is with sample data gathered
from the pumpkin pricing dataset that we will use throughout this
lesson. You have also visualized it using \texttt{ggplot2}.💪

Now you are ready to dive deeper into regression for ML. In this lesson,
you will learn more about two types of regression: \emph{basic linear
regression} and \emph{polynomial regression}, along with some of the
math underlying these techniques.

\begin{quote}
Throughout this curriculum, we assume minimal knowledge of math, and
seek to make it accessible for students coming from other fields, so
watch for notes, 🧮 callouts, diagrams, and other learning tools to aid
in comprehension.
\end{quote}

\hypertarget{preparation}{%
\paragraph{Preparation}\label{preparation}}

As a reminder, you are loading this data so as to ask questions of it.

\begin{itemize}
\item
  When is the best time to buy pumpkins?
\item
  What price can I expect of a case of miniature pumpkins?
\item
  Should I buy them in half-bushel baskets or by the 1 1/9 bushel box?
  Let's keep digging into this data.
\end{itemize}

In the previous lesson, you created a \texttt{tibble} (a modern
reimagining of the data frame) and populated it with part of the
original dataset, standardizing the pricing by the bushel. By doing
that, however, you were only able to gather about 400 data points and
only for the fall months. Maybe we can get a little more detail about
the nature of the data by cleaning it more? We'll see\ldots{} 🕵️‍♀️

\hypertarget{a-linear-regression-line}{%
\subsection{1. A linear regression
line}\label{a-linear-regression-line}}

As you learned in Lesson 1, the goal of a linear regression exercise is
to be able to plot a \emph{line} \emph{of} \emph{best fit} to:

\begin{itemize}
\item
  \textbf{Show variable relationships}. Show the relationship between
  variables
\item
  \textbf{Make predictions}. Make accurate predictions on where a new
  data point would fall in relationship to that line.
\end{itemize}

To draw this type of line, we use a statistical technique called
\textbf{Least-Squares Regression}. The term \texttt{least-squares} means
that all the data points surrounding the regression line are squared and
then added up. Ideally, that final sum is as small as possible, because
we want a low number of errors, or \texttt{least-squares}. As such, the
line of best fit is the line that gives us the lowest value for the sum
of the squared errors - hence the name \emph{least squares regression}.

We do so since we want to model a line that has the least cumulative
distance from all of our data points. We also square the terms before
adding them since we are concerned with its magnitude rather than its
direction.

\begin{quote}
\textbf{🧮 Show me the math}

This line, called the \emph{line of best fit} can be expressed by
\href{https://en.wikipedia.org/wiki/Simple_linear_regression}{an
equation}:

\begin{verbatim}
Y = a + bX
\end{verbatim}

\texttt{X} is the `\texttt{explanatory\ variable} or
\texttt{predictor}'. \texttt{Y} is the `\texttt{dependent\ variable} or
\texttt{outcome}'. The slope of the line is \texttt{b} and \texttt{a} is
the y-intercept, which refers to the value of \texttt{Y} when
\texttt{X\ =\ 0}.

\begin{figure}
\centering
\includegraphics[width=4.16667in,height=\textheight]{../../images/slope.png}
\caption{Infographic by Jen Looper}
\end{figure}

First, calculate the slope \texttt{b}.

In other words, and referring to our pumpkin data's original question:
``predict the price of a pumpkin per bushel by month'', \texttt{X} would
refer to the price and \texttt{Y} would refer to the month of sale.

\begin{figure}
\centering
\includegraphics{../../images/calculation.png}
\caption{Infographic by Jen Looper}
\end{figure}

Calculate the value of Y. If you're paying around \$4, it must be April!

The math that calculates the line must demonstrate the slope of the
line, which is also dependent on the intercept, or where \texttt{Y} is
situated when \texttt{X\ =\ 0}.

You can observe the method of calculation for these values on the
\href{https://www.mathsisfun.com/data/least-squares-regression.html}{Math
is Fun} web site. Also visit
\href{https://www.mathsisfun.com/data/least-squares-calculator.html}{this
Least-squares calculator} to watch how the numbers' values impact the
line.
\end{quote}

Not so scary, right? 🤓

\hypertarget{correlation}{%
\paragraph{Correlation}\label{correlation}}

One more term to understand is the \textbf{Correlation Coefficient}
between given X and Y variables. Using a scatterplot, you can quickly
visualize this coefficient. A plot with datapoints scattered in a neat
line have high correlation, but a plot with datapoints scattered
everywhere between X and Y have a low correlation.

A good linear regression model will be one that has a high (nearer to 1
than 0) Correlation Coefficient using the Least-Squares Regression
method with a line of regression.

\hypertarget{a-dance-with-data-creating-a-data-frame-that-will-be-used-for-modelling}{%
\subsection{\texorpdfstring{\textbf{2. A dance with data: creating a
data frame that will be used for
modelling}}{2. A dance with data: creating a data frame that will be used for modelling}}\label{a-dance-with-data-creating-a-data-frame-that-will-be-used-for-modelling}}

\begin{figure}
\centering
\includegraphics[width=7.29167in,height=\textheight]{../../images/janitor.jpg}
\caption{Artwork by @allison\_horst}
\end{figure}

Load up required libraries and dataset. Convert the data to a data frame
containing a subset of the data:

\begin{itemize}
\item
  Only get pumpkins priced by the bushel
\item
  Convert the date to a month
\item
  Calculate the price to be an average of high and low prices
\item
  Convert the price to reflect the pricing by bushel quantity
\end{itemize}

\begin{quote}
We covered these steps in the
\href{https://github.com/microsoft/ML-For-Beginners/blob/main/2-Regression/2-Data/solution/lesson_2-R.ipynb}{previous
lesson}.
\end{quote}

\begin{Shaded}
\begin{Highlighting}[]
\CommentTok{\# Load the core Tidyverse packages}
\FunctionTok{library}\NormalTok{(tidyverse)}
\FunctionTok{library}\NormalTok{(lubridate)}

\CommentTok{\# Import the pumpkins data}
\FunctionTok{setwd}\NormalTok{(}\StringTok{"C:/Users/HP/Documents/repositorio\_ML/ML{-}For{-}Beginners/2{-}Regression/data"}\NormalTok{)}
\NormalTok{pumpkins }\OtherTok{\textless{}{-}} \FunctionTok{read\_csv}\NormalTok{(}\AttributeTok{file =} \StringTok{"US{-}pumpkins.csv"}\NormalTok{)}


\CommentTok{\# Get a glimpse and dimensions of the data}
\FunctionTok{glimpse}\NormalTok{(pumpkins)}
\end{Highlighting}
\end{Shaded}

\begin{verbatim}
## Rows: 1,757
## Columns: 26
## $ `City Name`       <chr> "BALTIMORE", "BALTIMORE", "BALTIMORE", "BALTIMORE", ~
## $ Type              <chr> NA, NA, NA, NA, NA, NA, NA, NA, NA, NA, NA, NA, NA, ~
## $ Package           <chr> "24 inch bins", "24 inch bins", "24 inch bins", "24 ~
## $ Variety           <chr> NA, NA, "HOWDEN TYPE", "HOWDEN TYPE", "HOWDEN TYPE",~
## $ `Sub Variety`     <chr> NA, NA, NA, NA, NA, NA, NA, NA, NA, NA, NA, NA, NA, ~
## $ Grade             <lgl> NA, NA, NA, NA, NA, NA, NA, NA, NA, NA, NA, NA, NA, ~
## $ Date              <chr> "4/29/17", "5/6/17", "9/24/16", "9/24/16", "11/5/16"~
## $ `Low Price`       <dbl> 270, 270, 160, 160, 90, 90, 160, 160, 160, 160, 160,~
## $ `High Price`      <dbl> 280, 280, 160, 160, 100, 100, 170, 160, 170, 160, 17~
## $ `Mostly Low`      <dbl> 270, 270, 160, 160, 90, 90, 160, 160, 160, 160, 160,~
## $ `Mostly High`     <dbl> 280, 280, 160, 160, 100, 100, 170, 160, 170, 160, 17~
## $ Origin            <chr> "MARYLAND", "MARYLAND", "DELAWARE", "VIRGINIA", "MAR~
## $ `Origin District` <chr> NA, NA, NA, NA, NA, NA, NA, NA, NA, NA, NA, NA, NA, ~
## $ `Item Size`       <chr> "lge", "lge", "med", "med", "lge", "lge", "med", "lg~
## $ Color             <chr> NA, NA, "ORANGE", "ORANGE", "ORANGE", "ORANGE", "ORA~
## $ Environment       <lgl> NA, NA, NA, NA, NA, NA, NA, NA, NA, NA, NA, NA, NA, ~
## $ `Unit of Sale`    <chr> NA, NA, NA, NA, NA, NA, NA, NA, NA, NA, NA, NA, NA, ~
## $ Quality           <lgl> NA, NA, NA, NA, NA, NA, NA, NA, NA, NA, NA, NA, NA, ~
## $ Condition         <lgl> NA, NA, NA, NA, NA, NA, NA, NA, NA, NA, NA, NA, NA, ~
## $ Appearance        <lgl> NA, NA, NA, NA, NA, NA, NA, NA, NA, NA, NA, NA, NA, ~
## $ Storage           <lgl> NA, NA, NA, NA, NA, NA, NA, NA, NA, NA, NA, NA, NA, ~
## $ Crop              <lgl> NA, NA, NA, NA, NA, NA, NA, NA, NA, NA, NA, NA, NA, ~
## $ Repack            <chr> "E", "E", "N", "N", "N", "N", "N", "N", "N", "N", "N~
## $ `Trans Mode`      <lgl> NA, NA, NA, NA, NA, NA, NA, NA, NA, NA, NA, NA, NA, ~
## $ ...25             <lgl> NA, NA, NA, NA, NA, NA, NA, NA, NA, NA, NA, NA, NA, ~
## $ ...26             <chr> NA, NA, NA, NA, NA, NA, NA, NA, NA, NA, NA, NA, NA, ~
\end{verbatim}

\begin{Shaded}
\begin{Highlighting}[]
\CommentTok{\# Print the first 50 rows of the data set}
\NormalTok{pumpkins }\SpecialCharTok{\%\textgreater{}\%} 
  \FunctionTok{slice\_head}\NormalTok{(}\AttributeTok{n =} \DecValTok{5}\NormalTok{)}
\end{Highlighting}
\end{Shaded}

\begin{verbatim}
## # A tibble: 5 x 26
##   `City Name` Type  Package      Variety   `Sub Variety` Grade Date  `Low Price`
##   <chr>       <chr> <chr>        <chr>     <chr>         <lgl> <chr>       <dbl>
## 1 BALTIMORE   <NA>  24 inch bins <NA>      <NA>          NA    4/29~         270
## 2 BALTIMORE   <NA>  24 inch bins <NA>      <NA>          NA    5/6/~         270
## 3 BALTIMORE   <NA>  24 inch bins HOWDEN T~ <NA>          NA    9/24~         160
## 4 BALTIMORE   <NA>  24 inch bins HOWDEN T~ <NA>          NA    9/24~         160
## 5 BALTIMORE   <NA>  24 inch bins HOWDEN T~ <NA>          NA    11/5~          90
## # ... with 18 more variables: `High Price` <dbl>, `Mostly Low` <dbl>,
## #   `Mostly High` <dbl>, Origin <chr>, `Origin District` <chr>,
## #   `Item Size` <chr>, Color <chr>, Environment <lgl>, `Unit of Sale` <chr>,
## #   Quality <lgl>, Condition <lgl>, Appearance <lgl>, Storage <lgl>,
## #   Crop <lgl>, Repack <chr>, `Trans Mode` <lgl>, ...25 <lgl>, ...26 <chr>
\end{verbatim}

In the spirit of sheer adventure, let's explore the
\href{github.com/sfirke/janitor}{\texttt{janitor\ package}} that
provides simple functions for examining and cleaning dirty data. For
instance, let's take a look at the column names for our data:

\begin{Shaded}
\begin{Highlighting}[]
\CommentTok{\# Return column names}
\NormalTok{pumpkins }\SpecialCharTok{\%\textgreater{}\%} 
  \FunctionTok{names}\NormalTok{()}
\end{Highlighting}
\end{Shaded}

\begin{verbatim}
##  [1] "City Name"       "Type"            "Package"         "Variety"        
##  [5] "Sub Variety"     "Grade"           "Date"            "Low Price"      
##  [9] "High Price"      "Mostly Low"      "Mostly High"     "Origin"         
## [13] "Origin District" "Item Size"       "Color"           "Environment"    
## [17] "Unit of Sale"    "Quality"         "Condition"       "Appearance"     
## [21] "Storage"         "Crop"            "Repack"          "Trans Mode"     
## [25] "...25"           "...26"
\end{verbatim}

\begin{Shaded}
\begin{Highlighting}[]
\CommentTok{\# Clean names to the snake\_case convention}
\FunctionTok{library}\NormalTok{(janitor)}
\end{Highlighting}
\end{Shaded}

\begin{verbatim}
## Warning: package 'janitor' was built under R version 4.1.3
\end{verbatim}

\begin{verbatim}
## 
## Attaching package: 'janitor'
\end{verbatim}

\begin{verbatim}
## The following objects are masked from 'package:stats':
## 
##     chisq.test, fisher.test
\end{verbatim}

\begin{Shaded}
\begin{Highlighting}[]
\FunctionTok{library}\NormalTok{(snakecase)}
\end{Highlighting}
\end{Shaded}

\begin{verbatim}
## Warning: package 'snakecase' was built under R version 4.1.3
\end{verbatim}

\begin{Shaded}
\begin{Highlighting}[]
\NormalTok{pumpkins }\OtherTok{\textless{}{-}}\NormalTok{ pumpkins }\SpecialCharTok{\%\textgreater{}\%} 
  \FunctionTok{clean\_names}\NormalTok{(}\AttributeTok{case =} \StringTok{"snake"}\NormalTok{)}

\CommentTok{\# Return column names limpios, case snake es el caso por default}
\CommentTok{\#ej "\% successful (2009)" lo convierte percent\_successful\_2009}
\NormalTok{pumpkins }\SpecialCharTok{\%\textgreater{}\%} 
  \FunctionTok{names}\NormalTok{()}
\end{Highlighting}
\end{Shaded}

\begin{verbatim}
##  [1] "city_name"       "type"            "package"         "variety"        
##  [5] "sub_variety"     "grade"           "date"            "low_price"      
##  [9] "high_price"      "mostly_low"      "mostly_high"     "origin"         
## [13] "origin_district" "item_size"       "color"           "environment"    
## [17] "unit_of_sale"    "quality"         "condition"       "appearance"     
## [21] "storage"         "crop"            "repack"          "trans_mode"     
## [25] "x25"             "x26"
\end{verbatim}

Much tidyR 🧹! Now, a dance with the data using \texttt{dplyr} as in the
previous lesson! 💃

\begin{Shaded}
\begin{Highlighting}[]
\CommentTok{\# Select desired columns}
\NormalTok{pumpkins }\OtherTok{\textless{}{-}}\NormalTok{ pumpkins }\SpecialCharTok{\%\textgreater{}\%} 
  \FunctionTok{select}\NormalTok{(variety, city\_name, package, low\_price, high\_price, date)}



\CommentTok{\# Extract the month from the dates to a new column}
\NormalTok{pumpkins }\OtherTok{\textless{}{-}}\NormalTok{ pumpkins }\SpecialCharTok{\%\textgreater{}\%}
  \FunctionTok{mutate}\NormalTok{(}\AttributeTok{date =} \FunctionTok{mdy}\NormalTok{(date),}
         \AttributeTok{month =} \FunctionTok{month}\NormalTok{(date)) }\SpecialCharTok{\%\textgreater{}\%} 
  \FunctionTok{select}\NormalTok{(}\SpecialCharTok{{-}}\NormalTok{date)}



\CommentTok{\# Create a new column for average Price}
\NormalTok{pumpkins }\OtherTok{\textless{}{-}}\NormalTok{ pumpkins }\SpecialCharTok{\%\textgreater{}\%} 
  \FunctionTok{mutate}\NormalTok{(}\AttributeTok{price =}\NormalTok{ (low\_price }\SpecialCharTok{+}\NormalTok{ high\_price)}\SpecialCharTok{/}\DecValTok{2}\NormalTok{)}


\CommentTok{\# Retain only pumpkins with the string "bushel"}
\NormalTok{new\_pumpkins }\OtherTok{\textless{}{-}}\NormalTok{ pumpkins }\SpecialCharTok{\%\textgreater{}\%} 
  \FunctionTok{filter}\NormalTok{(}\FunctionTok{str\_detect}\NormalTok{(}\AttributeTok{string =}\NormalTok{ package, }\AttributeTok{pattern =} \StringTok{"bushel"}\NormalTok{))}


\CommentTok{\# Normalize the pricing so that you show the pricing per bushel, not per 1 1/9 or 1/2 bushel}
\NormalTok{new\_pumpkins }\OtherTok{\textless{}{-}}\NormalTok{ new\_pumpkins }\SpecialCharTok{\%\textgreater{}\%} 
  \FunctionTok{mutate}\NormalTok{(}\AttributeTok{price =} \FunctionTok{case\_when}\NormalTok{(}
    \FunctionTok{str\_detect}\NormalTok{(package, }\StringTok{"1 1/9"}\NormalTok{) }\SpecialCharTok{\textasciitilde{}}\NormalTok{ price}\SpecialCharTok{/}\NormalTok{(}\FloatTok{1.1}\NormalTok{),}
    \FunctionTok{str\_detect}\NormalTok{(package, }\StringTok{"1/2"}\NormalTok{) }\SpecialCharTok{\textasciitilde{}}\NormalTok{ price}\SpecialCharTok{*}\DecValTok{2}\NormalTok{,}
    \ConstantTok{TRUE} \SpecialCharTok{\textasciitilde{}}\NormalTok{ price))}

\CommentTok{\# Relocate column positions}
\NormalTok{new\_pumpkins }\OtherTok{\textless{}{-}}\NormalTok{ new\_pumpkins }\SpecialCharTok{\%\textgreater{}\%} 
  \FunctionTok{relocate}\NormalTok{(month, }\AttributeTok{.before =}\NormalTok{ variety)}


\CommentTok{\# Display the first 5 rows}
\NormalTok{new\_pumpkins }\SpecialCharTok{\%\textgreater{}\%} 
  \FunctionTok{slice\_head}\NormalTok{(}\AttributeTok{n =} \DecValTok{5}\NormalTok{)}
\end{Highlighting}
\end{Shaded}

\begin{verbatim}
## # A tibble: 5 x 7
##   month variety  city_name package              low_price high_price price
##   <dbl> <chr>    <chr>     <chr>                    <dbl>      <dbl> <dbl>
## 1     9 PIE TYPE BALTIMORE 1 1/9 bushel cartons        15         15  13.6
## 2     9 PIE TYPE BALTIMORE 1 1/9 bushel cartons        18         18  16.4
## 3    10 PIE TYPE BALTIMORE 1 1/9 bushel cartons        18         18  16.4
## 4    10 PIE TYPE BALTIMORE 1 1/9 bushel cartons        17         17  15.5
## 5    10 PIE TYPE BALTIMORE 1 1/9 bushel cartons        15         15  13.6
\end{verbatim}

Good job!👌 You now have a clean, tidy data set on which you can build
your new regression model!

Mind a scatter plot?

\begin{Shaded}
\begin{Highlighting}[]
\CommentTok{\# Set theme}
\FunctionTok{theme\_set}\NormalTok{(}\FunctionTok{theme\_light}\NormalTok{())}

\CommentTok{\# Make a scatter plot of month and price}
\NormalTok{new\_pumpkins }\SpecialCharTok{\%\textgreater{}\%} 
  \FunctionTok{ggplot}\NormalTok{(}\AttributeTok{mapping =} \FunctionTok{aes}\NormalTok{(}\AttributeTok{x =}\NormalTok{ month, }\AttributeTok{y =}\NormalTok{ price)) }\SpecialCharTok{+}
  \FunctionTok{geom\_point}\NormalTok{(}\AttributeTok{size =} \FloatTok{1.6}\NormalTok{)}
\end{Highlighting}
\end{Shaded}

\includegraphics{lesson_3_files/figure-latex/scatter_price_month-1.pdf}

A scatter plot reminds us that we only have month data from August
through December. We probably need more data to be able to draw
conclusions in a linear fashion.

Let's take a look at our modelling data again:

\begin{Shaded}
\begin{Highlighting}[]
\CommentTok{\# Display first 5 rows}
\NormalTok{new\_pumpkins }\SpecialCharTok{\%\textgreater{}\%} 
  \FunctionTok{slice\_head}\NormalTok{(}\AttributeTok{n =} \DecValTok{5}\NormalTok{)}
\end{Highlighting}
\end{Shaded}

\begin{verbatim}
## # A tibble: 5 x 7
##   month variety  city_name package              low_price high_price price
##   <dbl> <chr>    <chr>     <chr>                    <dbl>      <dbl> <dbl>
## 1     9 PIE TYPE BALTIMORE 1 1/9 bushel cartons        15         15  13.6
## 2     9 PIE TYPE BALTIMORE 1 1/9 bushel cartons        18         18  16.4
## 3    10 PIE TYPE BALTIMORE 1 1/9 bushel cartons        18         18  16.4
## 4    10 PIE TYPE BALTIMORE 1 1/9 bushel cartons        17         17  15.5
## 5    10 PIE TYPE BALTIMORE 1 1/9 bushel cartons        15         15  13.6
\end{verbatim}

What if we wanted to predict the \texttt{price} of a pumpkin based on
the \texttt{city} or \texttt{package} columns which are of type
character? Or even more simply, how could we find the correlation (which
requires both of its inputs to be numeric) between, say,
\texttt{package} and \texttt{price}? 🤷🤷

Machine learning models work best with numeric features rather than text
values, so you generally need to convert categorical features into
numeric representations.

This means that we have to find a way to reformat our predictors to make
them easier for a model to use effectively, a process known as
\texttt{feature\ engineering}.

\hypertarget{preprocessing-data-for-modelling-with-recipes}{%
\subsection{3. Preprocessing data for modelling with recipes
👩‍🍳👨‍🍳}\label{preprocessing-data-for-modelling-with-recipes}}

Activities that reformat predictor values to make them easier for a
model to use effectively has been termed \texttt{feature\ engineering}.

Different models have different preprocessing requirements. For
instance, least squares requires
\texttt{encoding\ categorical\ variables} such as month, variety and
city\_name. This simply involves \texttt{translating} a column with
\texttt{categorical\ values} into one or more \texttt{numeric\ columns}
that take the place of the original.

For example, suppose your data includes the following categorical
feature:

\begin{longtable}[]{@{}c@{}}
\toprule
city \\
\midrule
\endhead
Denver \\
Nairobi \\
Tokyo \\
\bottomrule
\end{longtable}

You can apply \emph{ordinal encoding} to substitute a unique integer
value for each category, like this:

\begin{longtable}[]{@{}c@{}}
\toprule
city \\
\midrule
\endhead
0 \\
1 \\
2 \\
\bottomrule
\end{longtable}

And that's what we'll do to our data!

In this section, we'll explore another amazing Tidymodels package:
\href{https://tidymodels.github.io/recipes/}{recipes} - which is
designed to help you preprocess your data \textbf{before} training your
model. At its core, a recipe is an object that defines what steps should
be applied to a data set in order to get it ready for modelling.

Now, let's create a recipe that prepares our data for modelling by
substituting a unique integer for all the observations in the predictor
columns:

\begin{Shaded}
\begin{Highlighting}[]
\FunctionTok{library}\NormalTok{(recipes)}
\end{Highlighting}
\end{Shaded}

\begin{verbatim}
## Warning: package 'recipes' was built under R version 4.1.3
\end{verbatim}

\begin{verbatim}
## 
## Attaching package: 'recipes'
\end{verbatim}

\begin{verbatim}
## The following object is masked from 'package:stringr':
## 
##     fixed
\end{verbatim}

\begin{verbatim}
## The following object is masked from 'package:stats':
## 
##     step
\end{verbatim}

\begin{Shaded}
\begin{Highlighting}[]
\CommentTok{\# Specify a recipe}
\NormalTok{pumpkins\_recipe }\OtherTok{\textless{}{-}} \FunctionTok{recipe}\NormalTok{(price }\SpecialCharTok{\textasciitilde{}}\NormalTok{ ., }\AttributeTok{data =}\NormalTok{ new\_pumpkins) }\SpecialCharTok{\%\textgreater{}\%} 
  \FunctionTok{step\_integer}\NormalTok{(}\FunctionTok{all\_predictors}\NormalTok{(), }\AttributeTok{zero\_based =} \ConstantTok{TRUE}\NormalTok{)}


\CommentTok{\# Print out the recipe}
\NormalTok{pumpkins\_recipe}
\end{Highlighting}
\end{Shaded}

\begin{verbatim}
## Recipe
## 
## Inputs:
## 
##       role #variables
##    outcome          1
##  predictor          6
## 
## Operations:
## 
## Integer encoding for all_predictors()
\end{verbatim}

Awesome! 👏 We just created our first recipe that specifies an outcome
(price) and its corresponding predictors and that all the predictor
columns should be encoded into a set of integers 🙌! Let's quickly break
it down:

\begin{itemize}
\item
  The call to \texttt{recipe()} with a formula tells the recipe the
  \emph{roles} of the variables using \texttt{new\_pumpkins} data as the
  reference. For instance the \texttt{price} column has been assigned an
  \texttt{outcome} role while the rest of the columns have been assigned
  a \texttt{predictor} role.
\item
  \texttt{step\_integer(all\_predictors(),\ zero\_based\ =\ TRUE)}
  specifies that all the predictors should be converted into a set of
  integers with the numbering starting at 0.
\end{itemize}

We are sure you may be having thoughts such as: ``This is so cool!! But
what if I needed to confirm that the recipes are doing exactly what I
expect them to do? 🤔''

That's an awesome thought! You see, once your recipe is defined, you can
estimate the parameters required to actually preprocess the data, and
then extract the processed data. You don't typically need to do this
when you use Tidymodels (we'll see the normal convention in just a
minute-\textgreater{} \texttt{workflows}) but it can come in handy when
you want to do some kind of sanity check for confirming that recipes are
doing what you expect.

For that, you'll need two more verbs: \texttt{prep()} and
\texttt{bake()} and as always, our little R friends by
\href{https://github.com/allisonhorst/stats-illustrations}{\texttt{Allison\ Horst}}
help you in understanding this better!

\href{https://recipes.tidymodels.org/reference/prep.html}{\texttt{prep()}}:
estimates the required parameters from a training set that can be later
applied to other data sets. For instance, for a given predictor column,
what observation will be assigned integer 0 or 1 or 2 etc

\href{https://recipes.tidymodels.org/reference/bake.html}{\texttt{bake()}}:
takes a prepped recipe and applies the operations to any data set.

That said, lets prep and bake our recipes to really confirm that under
the hood, the predictor columns will be first encoded before a model is
fit.

\begin{Shaded}
\begin{Highlighting}[]
\CommentTok{\# Prep the recipe}
\NormalTok{pumpkins\_prep }\OtherTok{\textless{}{-}} \FunctionTok{prep}\NormalTok{(pumpkins\_recipe)}

\CommentTok{\# Bake the recipe to extract a preprocessed new\_pumpkins data}
\NormalTok{baked\_pumpkins }\OtherTok{\textless{}{-}} \FunctionTok{bake}\NormalTok{(pumpkins\_prep, }\AttributeTok{new\_data =} \ConstantTok{NULL}\NormalTok{)}

\CommentTok{\# Print out the baked data set}
\NormalTok{baked\_pumpkins }\SpecialCharTok{\%\textgreater{}\%} 
  \FunctionTok{slice\_head}\NormalTok{(}\AttributeTok{n =} \DecValTok{10}\NormalTok{)}
\end{Highlighting}
\end{Shaded}

\begin{verbatim}
## # A tibble: 10 x 7
##    month variety city_name package low_price high_price price
##    <dbl>   <dbl>     <dbl>   <dbl>     <dbl>      <dbl> <dbl>
##  1     1       3         1       0         5          3  13.6
##  2     1       3         1       0        10          7  16.4
##  3     2       3         1       0        10          7  16.4
##  4     2       3         1       0         9          6  15.5
##  5     2       3         1       0         5          3  13.6
##  6     2       3         1       0        10          7  16.4
##  7     2       3         1       0         9          6  15.5
##  8     2       3         1       0         9          8  16.1
##  9     2       3         1       0         5          3  13.6
## 10     2       3         1       0         9          6  15.5
\end{verbatim}

Woo-hoo!🥳 The processed data \texttt{baked\_pumpkins} has all it's
predictors encoded confirming that indeed the preprocessing steps
defined as our recipe will work as expected. This makes it harder for
you to read but much more intelligible for Tidymodels! Take some time to
find out what observation has been mapped to a corresponding integer.

It is also worth mentioning that \texttt{baked\_pumpkins} is a data
frame that we can perform computations on.

For instance, let's try to find a good correlation between two points of
your data to potentially build a good predictive model. We'll use the
function \texttt{cor()} to do this. Type \texttt{?cor()} to find out
more about the function.

\begin{Shaded}
\begin{Highlighting}[]
\CommentTok{\# Find the correlation between the city\_name and the price}
\FunctionTok{cor}\NormalTok{(baked\_pumpkins}\SpecialCharTok{$}\NormalTok{city\_name, baked\_pumpkins}\SpecialCharTok{$}\NormalTok{price)}
\end{Highlighting}
\end{Shaded}

\begin{verbatim}
## [1] 0.3236397
\end{verbatim}

\begin{Shaded}
\begin{Highlighting}[]
\CommentTok{\# Find the correlation between the package and the price}
\FunctionTok{cor}\NormalTok{(baked\_pumpkins}\SpecialCharTok{$}\NormalTok{package, baked\_pumpkins}\SpecialCharTok{$}\NormalTok{price)}
\end{Highlighting}
\end{Shaded}

\begin{verbatim}
## [1] 0.6061713
\end{verbatim}

As it turns out, there's only weak correlation between the City and
Price. However there's a bit better correlation between the Package and
its Price. That makes sense, right? Normally, the bigger the produce
box, the higher the price.

While we are at it, let's also try and visualize a correlation matrix of
all the columns using the \texttt{corrplot} package.

\begin{Shaded}
\begin{Highlighting}[]
\CommentTok{\# Load the corrplot package}
\FunctionTok{library}\NormalTok{(corrplot)}
\end{Highlighting}
\end{Shaded}

\begin{verbatim}
## Warning: package 'corrplot' was built under R version 4.1.3
\end{verbatim}

\begin{verbatim}
## corrplot 0.92 loaded
\end{verbatim}

\begin{Shaded}
\begin{Highlighting}[]
\CommentTok{\# Obtain correlation matrix}
\NormalTok{corr\_mat }\OtherTok{\textless{}{-}} \FunctionTok{cor}\NormalTok{(baked\_pumpkins }\SpecialCharTok{\%\textgreater{}\%} 
                  \CommentTok{\# Drop columns that are not really informative}
                  \FunctionTok{select}\NormalTok{(}\SpecialCharTok{{-}}\FunctionTok{c}\NormalTok{(low\_price, high\_price)))}

\CommentTok{\# Make a correlation plot between the variables}
\FunctionTok{corrplot}\NormalTok{(corr\_mat, }\AttributeTok{method =} \StringTok{"shade"}\NormalTok{, }\AttributeTok{shade.col =} \ConstantTok{NA}\NormalTok{, }\AttributeTok{tl.col =} \StringTok{"black"}\NormalTok{, }\AttributeTok{tl.srt =} \DecValTok{45}\NormalTok{, }\AttributeTok{addCoef.col =} \StringTok{"black"}\NormalTok{, }\AttributeTok{cl.pos =} \StringTok{"n"}\NormalTok{, }\AttributeTok{order =} \StringTok{"original"}\NormalTok{)}
\end{Highlighting}
\end{Shaded}

\includegraphics{lesson_3_files/figure-latex/corrplot-1.pdf}

🤩🤩 Much better.

A good question to now ask of this data will be:
`\texttt{What\ price\ can\ I\ expect\ of\ a\ given\ pumpkin\ package?}'
Let's get right into it!

\begin{quote}
Note: When you \textbf{\texttt{bake()}} the prepped recipe
\textbf{\texttt{pumpkins\_prep}} with
\textbf{\texttt{new\_data\ =\ NULL}}, you extract the processed
(i.e.~encoded) training data. If you had another data set for example a
test set and would want to see how a recipe would pre-process it, you
would simply bake \textbf{\texttt{pumpkins\_prep}} with
\textbf{\texttt{new\_data\ =\ test\_set}}
\end{quote}

\hypertarget{build-a-linear-regression-model}{%
\subsection{4. Build a linear regression
model}\label{build-a-linear-regression-model}}

\begin{figure}
\centering
\includegraphics[width=8.33333in,height=\textheight]{../../images/linear-polynomial.png}
\caption{Infographic by Dasani Madipalli}
\end{figure}

Now that we have build a recipe, and actually confirmed that the data
will be pre-processed appropriately, let's now build a regression model
to answer the question:
\texttt{What\ price\ can\ I\ expect\ of\ a\ given\ pumpkin\ package?}

\hypertarget{train-a-linear-regression-model-using-the-training-set}{%
\paragraph{Train a linear regression model using the training
set}\label{train-a-linear-regression-model-using-the-training-set}}

As you may have already figured out, the column \emph{price} is the
\texttt{outcome} variable while the \emph{package} column is the
\texttt{predictor} variable.

To do this, we'll first split the data such that 80\% goes into training
and 20\% into test set, then define a recipe that will encode the
predictor column into a set of integers, then build a model
specification. We won't prep and bake our recipe since we already know
it will preprocess the data as expected.

\begin{Shaded}
\begin{Highlighting}[]
\FunctionTok{set.seed}\NormalTok{(}\DecValTok{2056}\NormalTok{)}
\FunctionTok{library}\NormalTok{(rsample)}
\end{Highlighting}
\end{Shaded}

\begin{verbatim}
## Warning: package 'rsample' was built under R version 4.1.3
\end{verbatim}

\begin{Shaded}
\begin{Highlighting}[]
\CommentTok{\# Split the data into training and test sets}
\NormalTok{pumpkins\_split }\OtherTok{\textless{}{-}}\NormalTok{ new\_pumpkins }\SpecialCharTok{\%\textgreater{}\%} 
  \FunctionTok{initial\_split}\NormalTok{(}\AttributeTok{prop =} \FloatTok{0.8}\NormalTok{)}


\CommentTok{\# Extract training and test data}
\NormalTok{pumpkins\_train }\OtherTok{\textless{}{-}} \FunctionTok{training}\NormalTok{(pumpkins\_split)}
\NormalTok{pumpkins\_test }\OtherTok{\textless{}{-}} \FunctionTok{testing}\NormalTok{(pumpkins\_split)}



\CommentTok{\# Create a recipe for preprocessing the data}
\NormalTok{lm\_pumpkins\_recipe }\OtherTok{\textless{}{-}} \FunctionTok{recipe}\NormalTok{(price }\SpecialCharTok{\textasciitilde{}}\NormalTok{ package, }\AttributeTok{data =}\NormalTok{ pumpkins\_train) }\SpecialCharTok{\%\textgreater{}\%} 
  \FunctionTok{step\_integer}\NormalTok{(}\FunctionTok{all\_predictors}\NormalTok{(), }\AttributeTok{zero\_based =} \ConstantTok{TRUE}\NormalTok{)}


\FunctionTok{library}\NormalTok{(parsnip)}
\CommentTok{\# Create a linear model specification}
\NormalTok{lm\_spec }\OtherTok{\textless{}{-}} \FunctionTok{linear\_reg}\NormalTok{() }\SpecialCharTok{\%\textgreater{}\%} 
  \FunctionTok{set\_engine}\NormalTok{(}\StringTok{"lm"}\NormalTok{) }\SpecialCharTok{\%\textgreater{}\%} 
  \FunctionTok{set\_mode}\NormalTok{(}\StringTok{"regression"}\NormalTok{)}
\end{Highlighting}
\end{Shaded}

Good job! Now that we have a recipe and a model specification, we need
to find a way of bundling them together into an object that will first
preprocess the data (prep+bake behind the scenes), fit the model on the
preprocessed data and also allow for potential post-processing
activities. How's that for your peace of mind!🤩

In Tidymodels, this convenient object is called a
\href{https://workflows.tidymodels.org/}{\texttt{workflow}} and
conveniently holds your modeling components! This is what we'd call
\emph{pipelines} in \emph{Python}.

So let's bundle everything up into a workflow!📦

\begin{Shaded}
\begin{Highlighting}[]
\FunctionTok{library}\NormalTok{(}\StringTok{"workflows"}\NormalTok{)}
\CommentTok{\# Hold modelling components in a workflow}
\NormalTok{lm\_wf }\OtherTok{\textless{}{-}} \FunctionTok{workflow}\NormalTok{() }\SpecialCharTok{\%\textgreater{}\%} 
  \FunctionTok{add\_recipe}\NormalTok{(lm\_pumpkins\_recipe) }\SpecialCharTok{\%\textgreater{}\%} 
  \FunctionTok{add\_model}\NormalTok{(lm\_spec)}

\CommentTok{\# Print out the workflow}
\NormalTok{lm\_wf}
\end{Highlighting}
\end{Shaded}

\begin{verbatim}
## == Workflow ====================================================================
## Preprocessor: Recipe
## Model: linear_reg()
## 
## -- Preprocessor ----------------------------------------------------------------
## 1 Recipe Step
## 
## * step_integer()
## 
## -- Model -----------------------------------------------------------------------
## Linear Regression Model Specification (regression)
## 
## Computational engine: lm
\end{verbatim}

👌 Into the bargain, a workflow can be fit/trained in much the same way a
model can.

\begin{Shaded}
\begin{Highlighting}[]
\CommentTok{\# Train the model}
\NormalTok{lm\_wf\_fit }\OtherTok{\textless{}{-}}\NormalTok{ lm\_wf }\SpecialCharTok{\%\textgreater{}\%} 
  \FunctionTok{fit}\NormalTok{(}\AttributeTok{data =}\NormalTok{ pumpkins\_train)}

\CommentTok{\# Print the model coefficients learned }
\NormalTok{lm\_wf\_fit}
\end{Highlighting}
\end{Shaded}

\begin{verbatim}
## == Workflow [trained] ==========================================================
## Preprocessor: Recipe
## Model: linear_reg()
## 
## -- Preprocessor ----------------------------------------------------------------
## 1 Recipe Step
## 
## * step_integer()
## 
## -- Model -----------------------------------------------------------------------
## 
## Call:
## stats::lm(formula = ..y ~ ., data = data)
## 
## Coefficients:
## (Intercept)      package  
##      19.977        4.884
\end{verbatim}

From the model output, we can see the coefficients learned during
training. They represent the coefficients of the line of best fit that
gives us the lowest overall error between the actual and predicted
variable.

\hypertarget{evaluate-model-performance-using-the-test-set}{%
\paragraph{Evaluate model performance using the test
set}\label{evaluate-model-performance-using-the-test-set}}

It's time to see how the model performed 📏! How do we do this?

Now that we've trained the model, we can use it to make predictions for
the test\_set using \texttt{parsnip::predict()}. Then we can compare
these predictions to the actual label values to evaluate how well (or
not!) the model is working.

Let's start with making predictions for the test set then bind the
columns to the test set.

\begin{Shaded}
\begin{Highlighting}[]
\CommentTok{\# Make predictions for the test set}
\NormalTok{predictions }\OtherTok{\textless{}{-}}\NormalTok{ lm\_wf\_fit }\SpecialCharTok{\%\textgreater{}\%} 
  \FunctionTok{predict}\NormalTok{(}\AttributeTok{new\_data =}\NormalTok{ pumpkins\_test)}


\CommentTok{\# Bind predictions to the test set}
\NormalTok{lm\_results }\OtherTok{\textless{}{-}}\NormalTok{ pumpkins\_test }\SpecialCharTok{\%\textgreater{}\%} 
  \FunctionTok{select}\NormalTok{(}\FunctionTok{c}\NormalTok{(package, price)) }\SpecialCharTok{\%\textgreater{}\%} 
  \FunctionTok{bind\_cols}\NormalTok{(predictions)}


\CommentTok{\# Print the first ten rows of the tibble}
\NormalTok{lm\_results }\SpecialCharTok{\%\textgreater{}\%} 
  \FunctionTok{slice\_head}\NormalTok{(}\AttributeTok{n =} \DecValTok{10}\NormalTok{)}
\end{Highlighting}
\end{Shaded}

\begin{verbatim}
## # A tibble: 10 x 3
##    package              price .pred
##    <chr>                <dbl> <dbl>
##  1 1 1/9 bushel cartons  13.6  20.0
##  2 1 1/9 bushel cartons  15.5  20.0
##  3 1 1/9 bushel cartons  16.4  20.0
##  4 1 1/9 bushel cartons  14.5  20.0
##  5 1 1/9 bushel cartons  14.5  20.0
##  6 1/2 bushel cartons    34    29.7
##  7 1/2 bushel cartons    30    29.7
##  8 1/2 bushel cartons    33    29.7
##  9 1/2 bushel cartons    30    29.7
## 10 1/2 bushel cartons    33    29.7
\end{verbatim}

Yes, you have just trained a model and used it to make predictions!🔮 Is
it any good, let's evaluate the model's performance!

In Tidymodels, we do this using \texttt{yardstick::metrics()}! For
linear regression, let's focus on the following metrics:

\begin{itemize}
\item
  \texttt{Root\ Mean\ Square\ Error\ (RMSE)}: The square root of the
  \href{https://en.wikipedia.org/wiki/Mean_squared_error}{MSE}. This
  yields an absolute metric in the same unit as the label (in this case,
  the price of a pumpkin). The smaller the value, the better the model
  (in a simplistic sense, it represents the average price by which the
  predictions are wrong!)
\item
  \texttt{Coefficient\ of\ Determination\ (usually\ known\ as\ R-squared\ or\ R2)}:
  A relative metric in which the higher the value, the better the fit of
  the model. In essence, this metric represents how much of the variance
  between predicted and actual label values the model is able to
  explain.
\end{itemize}

\begin{Shaded}
\begin{Highlighting}[]
\FunctionTok{library}\NormalTok{(yardstick)}
\end{Highlighting}
\end{Shaded}

\begin{verbatim}
## Warning: package 'yardstick' was built under R version 4.1.3
\end{verbatim}

\begin{verbatim}
## For binary classification, the first factor level is assumed to be the event.
## Use the argument `event_level = "second"` to alter this as needed.
\end{verbatim}

\begin{verbatim}
## 
## Attaching package: 'yardstick'
\end{verbatim}

\begin{verbatim}
## The following object is masked from 'package:readr':
## 
##     spec
\end{verbatim}

\begin{Shaded}
\begin{Highlighting}[]
\CommentTok{\# Evaluate performance of linear regression}
\FunctionTok{metrics}\NormalTok{(}\AttributeTok{data =}\NormalTok{ lm\_results,}
        \AttributeTok{truth =}\NormalTok{ price,}
        \AttributeTok{estimate =}\NormalTok{ .pred)}
\end{Highlighting}
\end{Shaded}

\begin{verbatim}
## # A tibble: 3 x 3
##   .metric .estimator .estimate
##   <chr>   <chr>          <dbl>
## 1 rmse    standard       7.56 
## 2 rsq     standard       0.449
## 3 mae     standard       5.81
\end{verbatim}

There goes the model performance. Let's see if we can get a better
indication by visualizing a scatter plot of the package and price then
use the predictions made to overlay a line of best fit.

This means we'll have to prep and bake the test set in order to encode
the package column then bind this to the predictions made by our model.

\begin{Shaded}
\begin{Highlighting}[]
\CommentTok{\# Encode package column}
\NormalTok{package\_encode }\OtherTok{\textless{}{-}}\NormalTok{ lm\_pumpkins\_recipe }\SpecialCharTok{\%\textgreater{}\%} 
  \FunctionTok{prep}\NormalTok{() }\SpecialCharTok{\%\textgreater{}\%} 
  \FunctionTok{bake}\NormalTok{(}\AttributeTok{new\_data =}\NormalTok{ pumpkins\_test) }\SpecialCharTok{\%\textgreater{}\%} 
  \FunctionTok{select}\NormalTok{(package)}


\CommentTok{\# Bind encoded package column to the results}
\NormalTok{lm\_results }\OtherTok{\textless{}{-}}\NormalTok{ lm\_results }\SpecialCharTok{\%\textgreater{}\%} 
  \FunctionTok{bind\_cols}\NormalTok{(package\_encode }\SpecialCharTok{\%\textgreater{}\%} 
              \FunctionTok{rename}\NormalTok{(}\AttributeTok{package\_integer =}\NormalTok{ package)) }\SpecialCharTok{\%\textgreater{}\%} 
  \FunctionTok{relocate}\NormalTok{(package\_integer, }\AttributeTok{.after =}\NormalTok{ package)}


\CommentTok{\# Print new results data frame}
\NormalTok{lm\_results }\SpecialCharTok{\%\textgreater{}\%} 
  \FunctionTok{slice\_head}\NormalTok{(}\AttributeTok{n =} \DecValTok{5}\NormalTok{)}
\end{Highlighting}
\end{Shaded}

\begin{verbatim}
## # A tibble: 5 x 4
##   package              package_integer price .pred
##   <chr>                          <dbl> <dbl> <dbl>
## 1 1 1/9 bushel cartons               0  13.6  20.0
## 2 1 1/9 bushel cartons               0  15.5  20.0
## 3 1 1/9 bushel cartons               0  16.4  20.0
## 4 1 1/9 bushel cartons               0  14.5  20.0
## 5 1 1/9 bushel cartons               0  14.5  20.0
\end{verbatim}

\begin{Shaded}
\begin{Highlighting}[]
\CommentTok{\# Make a scatter plot}
\NormalTok{lm\_results }\SpecialCharTok{\%\textgreater{}\%} 
  \FunctionTok{ggplot}\NormalTok{(}\AttributeTok{mapping =} \FunctionTok{aes}\NormalTok{(}\AttributeTok{x =}\NormalTok{ package\_integer, }\AttributeTok{y =}\NormalTok{ price)) }\SpecialCharTok{+}
  \FunctionTok{geom\_point}\NormalTok{(}\AttributeTok{size =} \FloatTok{1.6}\NormalTok{) }\SpecialCharTok{+}
  \CommentTok{\# Overlay a line of best fit}
  \FunctionTok{geom\_line}\NormalTok{(}\FunctionTok{aes}\NormalTok{(}\AttributeTok{y =}\NormalTok{ .pred), }\AttributeTok{color =} \StringTok{"orange"}\NormalTok{, }\AttributeTok{size =} \FloatTok{1.2}\NormalTok{) }\SpecialCharTok{+}
  \FunctionTok{xlab}\NormalTok{(}\StringTok{"package"}\NormalTok{)}
\end{Highlighting}
\end{Shaded}

\includegraphics{lesson_3_files/figure-latex/lm_plot-1.pdf}

Great! As you can see, the linear regression model does not really well
generalize the relationship between a package and its corresponding
price.

🎃 Congratulations, you just created a model that can help predict the
price of a few varieties of pumpkins. Your holiday pumpkin patch will be
beautiful. But you can probably create a better model!

\hypertarget{build-a-polynomial-regression-model}{%
\subsection{5. Build a polynomial regression
model}\label{build-a-polynomial-regression-model}}

\begin{figure}
\centering
\includegraphics[width=8.33333in,height=\textheight]{../../images/linear-polynomial.png}
\caption{Infographic by Dasani Madipalli}
\end{figure}

Sometimes our data may not have a linear relationship, but we still want
to predict an outcome. Polynomial regression can help us make
predictions for more complex non-linear relationships.

Take for instance the relationship between the package and price for our
pumpkins data set. While sometimes there's a linear relationship between
variables - the bigger the pumpkin in volume, the higher the price -
sometimes these relationships can't be plotted as a plane or straight
line.

\begin{quote}
✅ Here are \href{https://online.stat.psu.edu/stat501/lesson/9/9.8}{some
more examples} of data that could use polynomial regression

Take another look at the relationship between Variety to Price in the
previous plot. Does this scatterplot seem like it should necessarily be
analyzed by a straight line? Perhaps not. In this case, you can try
polynomial regression.

✅ Polynomials are mathematical expressions that might consist of one or
more variables and coefficients
\end{quote}

\hypertarget{train-a-polynomial-regression-model-using-the-training-set}{%
\paragraph{Train a polynomial regression model using the training
set}\label{train-a-polynomial-regression-model-using-the-training-set}}

Polynomial regression creates a \emph{curved line} to better fit
nonlinear data.

Let's see whether a polynomial model will perform better in making
predictions. We'll follow a somewhat similar procedure as we did before:

\begin{itemize}
\item
  Create a recipe that specifies the preprocessing steps that should be
  carried out on our data to get it ready for modelling i.e: encoding
  predictors and computing polynomials of degree \emph{n}
\item
  Build a model specification
\item
  Bundle the recipe and model specification into a workflow
\item
  Create a model by fitting the workflow
\item
  Evaluate how well the model performs on the test data
\end{itemize}

Let's get right into it!

\begin{Shaded}
\begin{Highlighting}[]
\CommentTok{\# Specify a recipe}
\NormalTok{poly\_pumpkins\_recipe }\OtherTok{\textless{}{-}}
  \FunctionTok{recipe}\NormalTok{(price }\SpecialCharTok{\textasciitilde{}}\NormalTok{ package, }\AttributeTok{data =}\NormalTok{ pumpkins\_train) }\SpecialCharTok{\%\textgreater{}\%}
  \FunctionTok{step\_integer}\NormalTok{(}\FunctionTok{all\_predictors}\NormalTok{(), }\AttributeTok{zero\_based =} \ConstantTok{TRUE}\NormalTok{) }\SpecialCharTok{\%\textgreater{}\%} 
  \FunctionTok{step\_poly}\NormalTok{(}\FunctionTok{all\_predictors}\NormalTok{(), }\AttributeTok{degree =} \DecValTok{4}\NormalTok{)}


\CommentTok{\# Create a model specification}
\NormalTok{poly\_spec }\OtherTok{\textless{}{-}} \FunctionTok{linear\_reg}\NormalTok{() }\SpecialCharTok{\%\textgreater{}\%} 
  \FunctionTok{set\_engine}\NormalTok{(}\StringTok{"lm"}\NormalTok{) }\SpecialCharTok{\%\textgreater{}\%} 
  \FunctionTok{set\_mode}\NormalTok{(}\StringTok{"regression"}\NormalTok{)}


\CommentTok{\# Bundle recipe and model spec into a workflow}
\NormalTok{poly\_wf }\OtherTok{\textless{}{-}} \FunctionTok{workflow}\NormalTok{() }\SpecialCharTok{\%\textgreater{}\%} 
  \FunctionTok{add\_recipe}\NormalTok{(poly\_pumpkins\_recipe) }\SpecialCharTok{\%\textgreater{}\%} 
  \FunctionTok{add\_model}\NormalTok{(poly\_spec)}


\CommentTok{\# Create a model}
\NormalTok{poly\_wf\_fit }\OtherTok{\textless{}{-}}\NormalTok{ poly\_wf }\SpecialCharTok{\%\textgreater{}\%} 
  \FunctionTok{fit}\NormalTok{(}\AttributeTok{data =}\NormalTok{ pumpkins\_train)}


\CommentTok{\# Print learned model coefficients}
\NormalTok{poly\_wf\_fit}
\end{Highlighting}
\end{Shaded}

\begin{verbatim}
## == Workflow [trained] ==========================================================
## Preprocessor: Recipe
## Model: linear_reg()
## 
## -- Preprocessor ----------------------------------------------------------------
## 2 Recipe Steps
## 
## * step_integer()
## * step_poly()
## 
## -- Model -----------------------------------------------------------------------
## 
## Call:
## stats::lm(formula = ..y ~ ., data = data)
## 
## Coefficients:
##    (Intercept)  package_poly_1  package_poly_2  package_poly_3  package_poly_4  
##         27.818         104.444        -113.001         -56.399           1.044
\end{verbatim}

\hypertarget{evaluate-model-performance}{%
\paragraph{Evaluate model
performance}\label{evaluate-model-performance}}

👏👏You've built a polynomial model let's make predictions on the test
set!

\begin{Shaded}
\begin{Highlighting}[]
\CommentTok{\# Make price predictions on test data}
\NormalTok{poly\_results }\OtherTok{\textless{}{-}}\NormalTok{ poly\_wf\_fit }\SpecialCharTok{\%\textgreater{}\%} \FunctionTok{predict}\NormalTok{(}\AttributeTok{new\_data =}\NormalTok{ pumpkins\_test) }\SpecialCharTok{\%\textgreater{}\%} 
  \FunctionTok{bind\_cols}\NormalTok{(pumpkins\_test }\SpecialCharTok{\%\textgreater{}\%} \FunctionTok{select}\NormalTok{(}\FunctionTok{c}\NormalTok{(package, price))) }\SpecialCharTok{\%\textgreater{}\%} 
  \FunctionTok{relocate}\NormalTok{(.pred, }\AttributeTok{.after =} \FunctionTok{last\_col}\NormalTok{())}


\CommentTok{\# Print the results}
\NormalTok{poly\_results }\SpecialCharTok{\%\textgreater{}\%} 
  \FunctionTok{slice\_head}\NormalTok{(}\AttributeTok{n =} \DecValTok{10}\NormalTok{)}
\end{Highlighting}
\end{Shaded}

\begin{verbatim}
## # A tibble: 10 x 3
##    package              price .pred
##    <chr>                <dbl> <dbl>
##  1 1 1/9 bushel cartons  13.6  15.7
##  2 1 1/9 bushel cartons  15.5  15.7
##  3 1 1/9 bushel cartons  16.4  15.7
##  4 1 1/9 bushel cartons  14.5  15.7
##  5 1 1/9 bushel cartons  14.5  15.7
##  6 1/2 bushel cartons    34    34.8
##  7 1/2 bushel cartons    30    34.8
##  8 1/2 bushel cartons    33    34.8
##  9 1/2 bushel cartons    30    34.8
## 10 1/2 bushel cartons    33    34.8
\end{verbatim}

Woo-hoo , let's evaluate how the model performed on the test\_set using
\texttt{yardstick::metrics()}.

\begin{Shaded}
\begin{Highlighting}[]
\FunctionTok{metrics}\NormalTok{(}\AttributeTok{data =}\NormalTok{ poly\_results, }\AttributeTok{truth =}\NormalTok{ price, }\AttributeTok{estimate =}\NormalTok{ .pred)}
\end{Highlighting}
\end{Shaded}

\begin{verbatim}
## # A tibble: 3 x 3
##   .metric .estimator .estimate
##   <chr>   <chr>          <dbl>
## 1 rmse    standard       3.59 
## 2 rsq     standard       0.885
## 3 mae     standard       2.77
\end{verbatim}

🤩🤩 Much better performance.

The \texttt{rmse} decreased from about 7. to about 3. an indication that
of a reduced error between the actual price and the predicted price. You
can \emph{loosely} interpret this as meaning that on average, incorrect
predictions are wrong by around \$3. The \texttt{rsq} increased from
about 0.4 to 0.8.

All these metrics indicate that the polynomial model performs way better
than the linear model. Good job!

Let's see if we can visualize this!

\begin{Shaded}
\begin{Highlighting}[]
\CommentTok{\# Bind encoded package column to the results}
\NormalTok{poly\_results }\OtherTok{\textless{}{-}}\NormalTok{ poly\_results }\SpecialCharTok{\%\textgreater{}\%} 
  \FunctionTok{bind\_cols}\NormalTok{(package\_encode }\SpecialCharTok{\%\textgreater{}\%} 
              \FunctionTok{rename}\NormalTok{(}\AttributeTok{package\_integer =}\NormalTok{ package)) }\SpecialCharTok{\%\textgreater{}\%} 
  \FunctionTok{relocate}\NormalTok{(package\_integer, }\AttributeTok{.after =}\NormalTok{ package)}


\CommentTok{\# Print new results data frame}
\NormalTok{poly\_results }\SpecialCharTok{\%\textgreater{}\%} 
  \FunctionTok{slice\_head}\NormalTok{(}\AttributeTok{n =} \DecValTok{5}\NormalTok{)}
\end{Highlighting}
\end{Shaded}

\begin{verbatim}
## # A tibble: 5 x 4
##   package              package_integer price .pred
##   <chr>                          <dbl> <dbl> <dbl>
## 1 1 1/9 bushel cartons               0  13.6  15.7
## 2 1 1/9 bushel cartons               0  15.5  15.7
## 3 1 1/9 bushel cartons               0  16.4  15.7
## 4 1 1/9 bushel cartons               0  14.5  15.7
## 5 1 1/9 bushel cartons               0  14.5  15.7
\end{verbatim}

\begin{Shaded}
\begin{Highlighting}[]
\CommentTok{\# Make a scatter plot}
\NormalTok{poly\_results }\SpecialCharTok{\%\textgreater{}\%} 
  \FunctionTok{ggplot}\NormalTok{(}\AttributeTok{mapping =} \FunctionTok{aes}\NormalTok{(}\AttributeTok{x =}\NormalTok{ package\_integer, }\AttributeTok{y =}\NormalTok{ price)) }\SpecialCharTok{+}
  \FunctionTok{geom\_point}\NormalTok{(}\AttributeTok{size =} \FloatTok{1.6}\NormalTok{) }\SpecialCharTok{+}
  \CommentTok{\# Overlay a line of best fit}
  \FunctionTok{geom\_line}\NormalTok{(}\FunctionTok{aes}\NormalTok{(}\AttributeTok{y =}\NormalTok{ .pred), }\AttributeTok{color =} \StringTok{"midnightblue"}\NormalTok{, }\AttributeTok{size =} \FloatTok{1.2}\NormalTok{) }\SpecialCharTok{+}
  \FunctionTok{xlab}\NormalTok{(}\StringTok{"package"}\NormalTok{)}
\end{Highlighting}
\end{Shaded}

\includegraphics{lesson_3_files/figure-latex/poly_viz-1.pdf}

You can see a curved line that fits your data better! 🤩

You can make this more smoother by passing a polynomial formula to
\texttt{geom\_smooth} like this:

\begin{Shaded}
\begin{Highlighting}[]
\CommentTok{\# Make a scatter plot}
\NormalTok{poly\_results }\SpecialCharTok{\%\textgreater{}\%} 
  \FunctionTok{ggplot}\NormalTok{(}\AttributeTok{mapping =} \FunctionTok{aes}\NormalTok{(}\AttributeTok{x =}\NormalTok{ package\_integer, }\AttributeTok{y =}\NormalTok{ price)) }\SpecialCharTok{+}
  \FunctionTok{geom\_point}\NormalTok{(}\AttributeTok{size =} \FloatTok{1.6}\NormalTok{) }\SpecialCharTok{+}
  \CommentTok{\# Overlay a line of best fit}
  \FunctionTok{geom\_smooth}\NormalTok{(}\AttributeTok{method =}\NormalTok{ lm, }\AttributeTok{formula =}\NormalTok{ y }\SpecialCharTok{\textasciitilde{}} \FunctionTok{poly}\NormalTok{(x, }\AttributeTok{degree =} \DecValTok{4}\NormalTok{), }\AttributeTok{color =} \StringTok{"midnightblue"}\NormalTok{, }\AttributeTok{size =} \FloatTok{1.2}\NormalTok{, }\AttributeTok{se =} \ConstantTok{FALSE}\NormalTok{) }\SpecialCharTok{+}
  \FunctionTok{xlab}\NormalTok{(}\StringTok{"package"}\NormalTok{)}
\end{Highlighting}
\end{Shaded}

\includegraphics{lesson_3_files/figure-latex/smooth curve-1.pdf}

Much like a smooth curve!🤩

Here's how you would make a new prediction:

\begin{Shaded}
\begin{Highlighting}[]
\CommentTok{\# Make a hypothetical data frame}
\NormalTok{hypo\_tibble }\OtherTok{\textless{}{-}} \FunctionTok{tibble}\NormalTok{(}\AttributeTok{package =} \StringTok{"bushel baskets"}\NormalTok{)}

\CommentTok{\# Make predictions using linear model}
\NormalTok{lm\_pred }\OtherTok{\textless{}{-}}\NormalTok{ lm\_wf\_fit }\SpecialCharTok{\%\textgreater{}\%} \FunctionTok{predict}\NormalTok{(}\AttributeTok{new\_data =}\NormalTok{ hypo\_tibble)}

\CommentTok{\# Make predictions using polynomial model}
\NormalTok{poly\_pred }\OtherTok{\textless{}{-}}\NormalTok{ poly\_wf\_fit }\SpecialCharTok{\%\textgreater{}\%} \FunctionTok{predict}\NormalTok{(}\AttributeTok{new\_data =}\NormalTok{ hypo\_tibble)}

\CommentTok{\# Return predictions in a list}
\FunctionTok{list}\NormalTok{(}\StringTok{"linear model prediction"} \OtherTok{=}\NormalTok{ lm\_pred, }
     \StringTok{"polynomial model prediction"} \OtherTok{=}\NormalTok{ poly\_pred)}
\end{Highlighting}
\end{Shaded}

\begin{verbatim}
## $`linear model prediction`
## # A tibble: 1 x 1
##   .pred
##   <dbl>
## 1  34.6
## 
## $`polynomial model prediction`
## # A tibble: 1 x 1
##   .pred
##   <dbl>
## 1  46.6
\end{verbatim}

The \texttt{polynomial\ model} prediction does make sense, given the
scatter plots of \texttt{price} and \texttt{package}! And, if this is a
better model than the previous one, looking at the same data, you need
to budget for these more expensive pumpkins!

🏆 Well done! You created two regression models in one lesson. In the
final section on regression, you will learn about logistic regression to
determine categories.

\hypertarget{challenge}{%
\subsection{\texorpdfstring{\textbf{🚀Challenge}}{🚀Challenge}}\label{challenge}}

Test several different variables in this notebook to see how correlation
corresponds to model accuracy.

\hypertarget{post-lecture-quiz}{%
\subsection{\texorpdfstring{\href{https://white-water-09ec41f0f.azurestaticapps.net/quiz/14/}{\textbf{Post-lecture
quiz}}}{Post-lecture quiz}}\label{post-lecture-quiz}}

\hypertarget{review-self-study}{%
\subsection{\texorpdfstring{\textbf{Review \& Self
Study}}{Review \& Self Study}}\label{review-self-study}}

In this lesson we learned about Linear Regression. There are other
important types of Regression. Read about Stepwise, Ridge, Lasso and
Elasticnet techniques. A good course to study to learn more is the
\href{https://online.stanford.edu/courses/sohs-ystatslearning-statistical-learning}{Stanford
Statistical Learning course}

If you want to learn more about how to use the amazing Tidymodels
framework, please check out the following resources:

\begin{itemize}
\item
  Tidymodels website: \href{https://www.tidymodels.org/start/}{Get
  started with Tidymodels}
\item
  Max Kuhn and Julia Silge, \href{https://www.tmwr.org/}{\emph{Tidy
  Modeling with R}}\emph{.}
\end{itemize}

\textbf{THANK YOU TO:}

\href{https://twitter.com/allison_horst?lang=en}{Allison Horst} for
creating the amazing illustrations that make R more welcoming and
engaging. Find more illustrations at her
\href{https://www.google.com/url?q=https://github.com/allisonhorst/stats-illustrations\&sa=D\&source=editors\&ust=1626380772530000\&usg=AOvVaw3zcfyCizFQZpkSLzxiiQEM}{gallery}.

\end{document}
